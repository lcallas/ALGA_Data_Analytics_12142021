% Options for packages loaded elsewhere
\PassOptionsToPackage{unicode}{hyperref}
\PassOptionsToPackage{hyphens}{url}
%
\documentclass[
]{article}
\usepackage{lmodern}
\usepackage{amsmath}
\usepackage{ifxetex,ifluatex}
\ifnum 0\ifxetex 1\fi\ifluatex 1\fi=0 % if pdftex
  \usepackage[T1]{fontenc}
  \usepackage[utf8]{inputenc}
  \usepackage{textcomp} % provide euro and other symbols
  \usepackage{amssymb}
\else % if luatex or xetex
  \usepackage{unicode-math}
  \defaultfontfeatures{Scale=MatchLowercase}
  \defaultfontfeatures[\rmfamily]{Ligatures=TeX,Scale=1}
\fi
% Use upquote if available, for straight quotes in verbatim environments
\IfFileExists{upquote.sty}{\usepackage{upquote}}{}
\IfFileExists{microtype.sty}{% use microtype if available
  \usepackage[]{microtype}
  \UseMicrotypeSet[protrusion]{basicmath} % disable protrusion for tt fonts
}{}
\makeatletter
\@ifundefined{KOMAClassName}{% if non-KOMA class
  \IfFileExists{parskip.sty}{%
    \usepackage{parskip}
  }{% else
    \setlength{\parindent}{0pt}
    \setlength{\parskip}{6pt plus 2pt minus 1pt}}
}{% if KOMA class
  \KOMAoptions{parskip=half}}
\makeatother
\usepackage{xcolor}
\IfFileExists{xurl.sty}{\usepackage{xurl}}{} % add URL line breaks if available
\IfFileExists{bookmark.sty}{\usepackage{bookmark}}{\usepackage{hyperref}}
\hypersetup{
  pdftitle={How to install and use R-Studio},
  pdfauthor={Kyla McConnell \& Julia Mueller},
  hidelinks,
  pdfcreator={LaTeX via pandoc}}
\urlstyle{same} % disable monospaced font for URLs
\usepackage[margin=1in]{geometry}
\usepackage{longtable,booktabs}
\usepackage{calc} % for calculating minipage widths
% Correct order of tables after \paragraph or \subparagraph
\usepackage{etoolbox}
\makeatletter
\patchcmd\longtable{\par}{\if@noskipsec\mbox{}\fi\par}{}{}
\makeatother
% Allow footnotes in longtable head/foot
\IfFileExists{footnotehyper.sty}{\usepackage{footnotehyper}}{\usepackage{footnote}}
\makesavenoteenv{longtable}
\usepackage{graphicx}
\makeatletter
\def\maxwidth{\ifdim\Gin@nat@width>\linewidth\linewidth\else\Gin@nat@width\fi}
\def\maxheight{\ifdim\Gin@nat@height>\textheight\textheight\else\Gin@nat@height\fi}
\makeatother
% Scale images if necessary, so that they will not overflow the page
% margins by default, and it is still possible to overwrite the defaults
% using explicit options in \includegraphics[width, height, ...]{}
\setkeys{Gin}{width=\maxwidth,height=\maxheight,keepaspectratio}
% Set default figure placement to htbp
\makeatletter
\def\fps@figure{htbp}
\makeatother
\setlength{\emergencystretch}{3em} % prevent overfull lines
\providecommand{\tightlist}{%
  \setlength{\itemsep}{0pt}\setlength{\parskip}{0pt}}
\setcounter{secnumdepth}{-\maxdimen} % remove section numbering
\ifluatex
  \usepackage{selnolig}  % disable illegal ligatures
\fi

\title{How to install and use R-Studio}
\author{Kyla McConnell \& Julia Mueller}
\date{}

\begin{document}
\maketitle

\hypertarget{if-youre-installing-r-for-the-first-time}{%
\subsection{If you're installing R for the first
time}\label{if-youre-installing-r-for-the-first-time}}

R is a programming language that can be written directly into your
computer's console/terminal. However, almost all users of R pair it with
the program R-Studio, an integrated development environment (IDE) that
offers a ton of features that will make your life much easier!

Download R here (if in Germany):
\url{https://ftp.gwdg.de/pub/misc/cran/} Or select a closer location
(``mirror'') here to optimize performance:
\url{https://ftp.gwdg.de/pub/misc/cran/}

Then download R-Studio (Desktop) here:
\url{https://rstudio.com/products/rstudio/download/}

More installation info here if you have trouble:
\url{https://rstudio-education.github.io/hopr/starting.html}

Windows users -- you might be prompted to download a program called
RTools to use the package panel in R Studio. You can do this directly
from the RStudio/CRAN webpage here:
\url{https://cran.rstudio.com/bin/windows/Rtools/}

Mac users -- a program called XQuartz is needed for viewing some plots.
This might not come up as a problem for you until later, but to be safe,
you can also go ahead and install it now: \url{https://www.xquartz.org/}

Note: You must install both R AND R-Studio in order to use R-Studio for
the rest of this tutorial.

\hypertarget{if-you-need-to-update-r-and-r-studio}{%
\subsection{If you need to update R and
R-Studio}\label{if-you-need-to-update-r-and-r-studio}}

If you want to update R, you can simply download the newest version
using the link above. Note, however, that you will likely have to
reinstall all packages. This is still worth doing every so often so that
you don't get too far behind, because new packages might not work on old
R versions (but old packages might not work on new R versions).

You can also update R Studio by redownloading it with the link above.

If you're on Windows, you can also use the installr package, here is
some more info:
\url{https://www.r-statistics.com/2013/03/updating-r-from-r-on-windows-using-the-installr-package/}

For Mac users, the

\hypertarget{about-r-studio}{%
\subsection{About R-Studio}\label{about-r-studio}}

R-Studio can make your R journey much easier! With R-Studio, you go from
a programming language of text-only to a whole program, not too unlike
Microsoft Word in comparison to text files.

In general, you have four panels:

Top-left: your scripting panel, in which you can write and save code.
This may only open when you open a new file (File -\textgreater{} New
File -\textgreater{} R Script / R-Markdown, we'll talk about the
difference below)

Bottom-left: the console panel, for input/output that won't be saved.

Top-right: the environment panel, which generally can show you
dataframes you are working with as well as other information you've
saved to variables (and sometimes has other features too, depending on
what you're working with)

Bottom-right: install and update packages, preview plots, read help
files, and some other features

\includegraphics{img/rstudio_panels.png} Image from
\url{https://ourcodingclub.github.io/tutorials/intro-to-r/}

\hypertarget{r-scripts-vs.-r-markdowns}{%
\subsection{R scripts vs.~R-Markdowns}\label{r-scripts-vs.-r-markdowns}}

There are two types of file that you can use with R-Studio: a simple
script, and a more feature rich R-Markdown document like this one.

Both of these file types allow you to save code to continue working on
t, re-run it later, or have as a record of your analysis steps.

\hypertarget{r-script-.r}{%
\subsubsection{R script (.R)}\label{r-script-.r}}

R scripts are the most simple files for saving your R code. They consist
only of code.

To open a new script, select File -\textgreater{} New File
-\textgreater{} R Script

In this type of file, you write lines of code in the order they are
meant to be run, and can run each line by clicking that line and
pressing Control + R (on Windows) or Cmd + Enter (on Mac)

You can try out this type of script by typing some mathematical
operations in the script and running those lines. The output will show
below, in your console panel.

All text in an R script is read as code. To leave comments for yourself
or to indicate that a certain line shouldn't be run as code, use a
hashtag (\#) at the beginning of that line.

\hypertarget{r-markdown-file-.rmd}{%
\subsubsection{R-Markdown file (.Rmd)}\label{r-markdown-file-.rmd}}

R-Markdown files are more complex files, that contain both code and
text. They can also include images and some basic formatting.

To open a new R-Markdown file, select File -\textgreater{} New File
-\textgreater{} R Markdown This will open a pop-up where you can give
your file a title (you can also change this later).

When you open a new R-Markdown file, you will be shown a basic template.
You can see a header at the top, called the YAML information for that
file.

For example:

\begin{longtable}[]{@{}l@{}}
\toprule
\endhead
title: ``Intro to R''\tabularnewline
author: ``Kyla McConnell''\tabularnewline
date: ``12/21/2020''\tabularnewline
output: html\_document\tabularnewline
\bottomrule
\end{longtable}

Changing this information will change what is displayed if you save the
file to a final format, which is denoted in the ``output'' line. It's
not super important for basic applications of R-Markdown but can be
customized more for more advanced uses.

If you type into general whitespace in a normal R-file, this will be
interpreted as code. In R-Markdown, however, text in whitespace is
interpreted as plain text. You can use this to keep notes for yourself
when learning something, or leave a record of your decisions in an
analysis.

To add code to R-Markdown, you use code chunks that you can either type
out (three backticks and \{r\} at the beginning and three backticks at
the end) or add by clicking the ``Insert'' button in the top bar and
selecting R.

To run code in these blocks, click the green arrow at the top right
corner of the box. The arrow with the green line under it will run all
code blocks above the current one, but will not run the current block
itself.

The output of the code block will show directly below it (not in the
console, like in a standard R script.) Often, this output will be
proceeded by a number in brackets -- this just helps you keep track of
how many numbers are output if they are on more than one line. Most of
the time, you can safely ignore this.

\end{document}
